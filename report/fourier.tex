\section{The Fourier Transform\label{sec:fourier}}
\subsection{Definition}
The Fourier Transform (FT) is a transform
that decomposes a function from the space/time domain into the frequency domain. %TODO: Domain?
If we view a function $g$ as an infinite-dimensional vector,
then FT transforms $g$ from a representation in the standard basis
to one in a basis of complex sinusoids.
It can be defined as follows:
%
\begin{equation}
     \hat g(f) = \int_{-\infty}^{\infty} g(x) \cdot e^{-i 2 \pi f x} dx
\end{equation}
%
An excellent visual representation of the transform is shown by \cite{3blue} in video form.
The gist of it is that for all frequencies $f$,
$g$ is wound around the unit circle in the complex plane with periodicity corresponding to $f$,
and the $\hat g(f)$ is the ``average'' value is given by the integral.
Any frequencies present in $g$ will be represented as a deviation from zero in $\hat g(f)$.
%
% TODO: Figure or remove 3blue reference

The Fourier inversion theorem states that,
for certain functions,
the inverse of the Fourier transform can be obtained by:
\begin{equation}
     g(x) = \int_{-\infty}^{\infty} \hat g(f) \cdot e^{i 2 \pi f x} df
\end{equation}

\subsection{Discrete Fourier Transform}
In real-world applications, we usually aren't working with continuous functions.
Typically we have a sequence of uniformly spaced samples approximating a recorded signal,
and we wish to compute something FT-like for this.
We cannot compute the FT of such a signal,
but we can approximate it using the Discrete Fourier Transform (DFT):
%
\begin{equation}
    X(k) = \sum_{n = 0}^{N - 1} x(n) \cdot e^{-i 2 \pi kn / N}
\end{equation}
%
The DFT is also an invertible, linear transformation.
The Inverse Discrete Fourier Transform (IDFT) is defined as:
\begin{equation}
    x(n) = \sum_{k = 0}^{N - 1} X(k) \cdot e^{i 2 \pi kn / N}
\end{equation}
%
We wish to write an efficient reversible algorithm
which computes the DFT when run in the forward direction
and computes the IDFT when run backwards.
