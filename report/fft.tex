\section{Fast Fourier Transform}

The cost of naively computing the DFT of an input is $O(N^2)$.
Luckily, by exploiting certain inherent redundancies in the computation,
we can shrink this to $O(N \log N)$.

\subsection{Twiddle factor}

To simplify the representation and computation of the DFT,
we define what is often referred to as the \textit{twiddle factor}:

\begin{equation}
    W_N = e^{-2 \pi i / N}
\end{equation}

Using this, we can then rephrase the DFT as:
\begin{equation}
    X(k) = \sum_{n = 0}^{N - 1} W_N^{nk} x(n)~~,~~k \in [0; N)
\end{equation}

\begin{figure}
    \centering
    \begin{tikzpicture}

    \end{tikzpicture}
    \caption{The twiddle factor in the complex plane.\label{fig:twid}}
\end{figure}

The twiddle factor essentially defines a rotation of a number relative to zero in the complex plane.
By multiplying a number $x$ by $W_N$, it is rotated by $1/N$'th of a full rotation.
By extension, $W_N^{k}$ corresponds to $k/N$'ths of a full rotation.
%TODO: Figure

$W_N^k$ itself just traverses the unit circle in the complex plane,
making a full rotation for every $N$ added to $k$.
This results in some inherent properties which we can exploit:
\begin{align}
    \textbf{Periodicity:} &~~W^k_N = W^{k + N}_N\\
    \textbf{Symmetry:}    &~~W^{k + N/2}_N = -W^k_N\\
    \textbf{Granularity:} &~~W_N^{ck} = W_{N/c}^{k}
\end{align}

\subsection{Factorization}

There are numerous different ways in which the DFT sum can be factorized.
They all have the same $O(N \log N)$ time complexity,
but differ in how many operations are concretely performed.
In the interest of simplicity,
we choose a factorization that is easy to understand rather than the most efficient one.
% TODO: Does this factorization have a name?

We can split the sum that defines the DFT into two sums of even and odd inputs:
\begin{align}
    X(k) &= \sum_{n = 0}^{N - 1} W_N^{nk} x(n) \\
    &= \sum_{n = 0}^{N/2 - 1} W_N^{2nk} x(2n) + \sum_{n = 0}^{N/2 - 1} W_N^{(2n + 1)k} x(2n + 1) \\
    &= \sum_{n = 0}^{N/2 - 1} W_N^{2nk} x(2n) + W_N^k \sum_{n = 0}^{N/2 - 1} W_N^{2nk} x(2n + 1) %\\
\end{align}
Let's define the even and odd inputs as lists:
\begin{align}
    x_\textit{even} &= \langle x_{2n} ~|~ 0 \leq n < N/2 \rangle\\
    x_\textit{odd}  &= \langle x_{2n + 1} ~|~ 0 \leq n < N/2 \rangle
\end{align}
Recalling the granularity property of the twiddle factor,
we can see that the DFT of these lists match the two sums that we just split $X(k)$ into:
\begin{align}
    X_\textit{even}(k) &= \sum_{n = 0}^{N/2 - 1} W_{N/2}^{nk} x_\textit{even}(n) = \sum_{n = 0}^{N/2 - 1} W_{N}^{2nk} x(2n) \\
    X_\textit{odd}(k) &= \sum_{n = 0}^{N/2 - 1} W_{N/2}^{nk} x_\textit{odd}(n) = \sum_{n = 0}^{N/2 - 1} W_{N}^{2nk} x(2n + 1)
\end{align}
We can then substitute to define $X(k)$ through those DFTs instead:
\begin{equation}
    X(k) = X_\textit{even}(k) + W_N^k X_\textit{odd}(k) ~,~ k \in [0;N/2)
\end{equation}
This only allows us to define $X(k)$ for $k \in [0;N/2)$ however.
To get the rest, we exploit the periodicity of the twiddle factor.
Since $W_{N/2}^{k + N/2} = W_{N/2}^k$, we also have $X_\textit{even}(k + N/2) = X_\textit{even}(k)$.
Thus we can wrap around for $k \in [N/2;N)$ and use $k' = k - N/2$ in the stead of $k$.

For the upper values of $k$, we can save more work.
From the symmetry property we get $W_N^k = W_N^{k - N/2} = -W_N^{k}$.
This gives us the final formulation of our factorization:
\begin{equation}
    X(k) =
    \begin{cases}
        X_\textit{even}(k) + W_N^k X_\textit{odd}(k) &\text{if}~k \in [0;N/2) \\
        X_\textit{even}(k') - W_N^{k'} X_\textit{odd}(k') &\text{if}~k \in [N/2;N) ~\text{where}~k' = k - N/2
    \end{cases}
\end{equation}

\subsubsection{Complexity}

Computing $X(k)$ from $X_\textit{even}(k)$ and $X_\textit{odd}(k)$
only requires a constant amount of work.
Doing this for all $k \in [0;N)$ takes $O(N)$ time.
Additionally, we can recursively compute the DFT of $x_\textit{even}$ and $x_\textit{odd}$
using the same strategy.
We can thus prove the complexity bound with a recursive formula:
\begin{equation}
    T(N) = O(N) + 2T(N/2) = O(N \log N)
\end{equation}
