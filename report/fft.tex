\section{Fast Fourier Transform}

The time cost complexity of naively computing the DFT of a signal is $O(N^2)$.
Luckily, by exploiting certain inherent properties of the transform,
we can reduce this to $O(N \log N)$ while greatly reducing the needed number of multiplications.

\subsection{Twiddle factor}

To simplify the representation and computation of the DFT,
we define the following constant:

\begin{equation}
    W_N = e^{-2 \pi i / N}
\end{equation}

This is often referred to as the \textit{twiddle factor}.
When the parameter $N$ isn't specified,
it should be inferred contextually.

Using this, we can then rephrase the DFT as:
\begin{equation}
    X(k) = \sum_{n = 0}^{N - 1} W_N^{nk} x(n)~~,~~k \in [0; N)
\end{equation}

\subsection{Periodicity and symmetry}

Some properties of the twiddle factor are exploited in order to minimize the amount of operations
and develop an $O(n \log n)$ algorithm for computing the DFT.
In order to understand these,
we must take a closer look at the twiddle factor and what it actually means.

We can express this succinctly by:
\begin{equation}
    W_N^{k} = W_N^{k \mod N}
\end{equation}

From this we get two different properties.
The first, periodicity, is essential to achieving an $O(n \log n)$ time complexity.

\begin{equation}
    W^k_N = W^{k + N}_N
\end{equation}

The second property, symmetry,
will allow us to further reduce the amount of necessary multiplications.

\begin{equation}
    W^{k + N/2}_N = -W^k_N
\end{equation}

A third property can be exploited to define FFT as a fully recursive algorithm:

\begin{equation}
    W_N^{ck} = W_{N/c}^{k}
\end{equation}

\subsection{Factorization}

There are numerous different ways in which the DFT sum can be factorized.
They all have the same $O(N \log N)$ time complexity,
but differ in how many operations are concretely performed.
In the interest of simplicity,
we choose a factorization that is easy to understand rather than an efficient one.
% TODO: Does this factorization have a name?

%Recalling the third property of the twiddle factor,
We can split the sum that defines the DFT into two sums of even and odd inputs:
\begin{align}
    X(k) &= \sum_{n = 0}^{N - 1} W_N^{nk} x(n) \\
    &= \sum_{n = 0}^{N/2 - 1} W_N^{2nk} x(2n) + \sum_{n = 0}^{N/2 - 1} W_N^{(2n + 1)k} x(2n + 1) \\
    &= \sum_{n = 0}^{N/2 - 1} W_N^{2nk} x(2n) + W_N^k \sum_{n = 0}^{N/2 - 1} W_N^{2nk} x(2n + 1) %\\
    %&= \sum_{n = 0}^{N/2 - 1} W_{N/2}^{nk} x(2n) + W_N^k \sum_{n = 0}^{N/2 - 1} W_{N/2}^{nk} x(2n + 1)
\end{align}
Let's define those even and odd inputs as lists:
\begin{align}
    x_\textit{even} &= \langle x_{2n} ~|~ 0 \leq n < N/2 \rangle\\
    x_\textit{odd}  &= \langle x_{2n + 1} ~|~ 0 \leq n < N/2 \rangle
\end{align}
The DFTs of these two lists are almost equivalent to the DFT of $x$.
Recall the third property property of the twiddle factor:
\begin{align}
    X_\textit{even}(k) &= \sum_{n = 0}^{N/2 - 1} W_{N/2}^{nk} x_\textit{even}(n) = \sum_{n = 0}^{N/2 - 1} W_{N}^{2nk} x(2n) \\
    X_\textit{odd}(k) &= \sum_{n = 0}^{N/2 - 1} W_{N/2}^{nk} x_\textit{odd}(n) = \sum_{n = 0}^{N/2 - 1} W_{N}^{2nk} x(2n + 1)
\end{align}
We can define the DFT of $x$ through the DFT of the even/odd lists:
\begin{equation}
    X(k) = X_\textit{even}(k) + W_N^k X_\textit{odd}(k) ~,~ k \in [0;N/2)
\end{equation}
There is one problem, however,
as the DFTs of the splits are only defined for $k \in [0;N/2)$.

\begin{equation}
    X(k) =
    \begin{cases}
        X_\textit{even}(k) + W_N^k X_\textit{odd}(k) &\text{if}~k < N\\
        X_\textit{even}(N - k) - W_N^{k'} X_\textit{odd}(k') &\text{if}~k \geq N ~\text{where}~k' = k - N/2
    \end{cases}
\end{equation}
