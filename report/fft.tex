\section{Fast Fourier Transform}

The time cost complexity of naively computing the DFT of a signal is $O(N^2)$.
Luckily, by exploiting certain inherent properties of the transform,
we can reduce this to $O(N \log N)$ while greatly reducing the needed number of multiplications.

\subsection{Twiddle factor}

To simplify the representation and computation of the DFT,
we define the following constant:

\begin{equation}
    W_N = e^{-2 \pi i / N}
\end{equation}

This is often referred to as the \textit{twiddle factor}.
When the parameter $N$ isn't specified,
it should be inferred contextually.

Using this, we can then rephrase the DFT as:
\begin{equation}
    X(k) = \sum_{n = 0}^{N - 1} W_N^{kn}~~,~~k \in [0; N - 1]
\end{equation}

\subsection{Periodicity and symmetry}

Two properties of the twiddle factor are used in order to minimize the amount of operations
and develop an $O(n \log n)$ algorithm for computing the DFT.
In order to understand these,
we must take a closer look at the twiddle factor and what it actually means.

We can express this succinctly by:
\begin{equation}
    W_N^{k} = W_N^{k \mod N}
\end{equation}

From this we get two different properties.
The first, periodicity, is essential to achieving an $O(n \log n)$ time complexity.

\begin{equation}
    W^k_N = W^{k + N}_N
\end{equation}

The second property, symmetry,
will allow us to further reduce the amount of necessary multiplications.

\begin{equation}
    W^{k + N/2}_N = -W^k_N
\end{equation}

\subsection{Factorization(s)}

\subsubsection{Cooley-tukey}

\subsubsection{Split-radix}

