\section{Conclusion}
In this report, we have explored the Fourier Transform and its discrete fixed-time sibling, the
Discrete Fourier Transform.
We have seen how the DFT sum can be factorized into a recursive formulation,
allowing us to develop an $O(N \log N)$-time algorithm for computing it,
called the Fast Fourier Transform.
We have shown how, using even-odd scrambling, fixpoint arithmetic,
and lifting schemes, the FFT can be computed in-place and reversibly.
Finally, we have implemented this FFT in the reversible programming language Janus,
demonstrating that our algorithm is reversible in practice.

Further work might explore the challenge of making the transform bijective,
with the goal of deciding whether or not this is possible.
It would also be interesting to explore the possibility of implementing
the Discrete Cosine Transform
\footnote{\url{https://en.wikipedia.org/wiki/Discrete_cosine_transform}}
in a reversible language.
