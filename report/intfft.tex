\section{Reversible Integer Approximation}

From a theoretical perspective,
the Fourier Transform is straightforwardly reversible:
\begin{equation}
    g(k) = \int_{-\infty}^{\infty} \hat g(x) \cdot e^{2 \pi i k n / N} dx
\end{equation}

% TODO: Get into orthogonal matrices and complex conjugates.

Similarly, the inverse DFT (IDFT) can be defined as:
\begin{equation}
    x(n) = \sum_{k = 0}^{N - 1} X(n) \cdot e^{2 \pi i kn / N}
\end{equation}
This can factorized with the exact same methods used for the forward transform.

We know that FFT doesn't inherently destroy any information.
While this may be the case,
there are several challenges to using this invertibility in practice.

The first issue is resolution.
FFT performs many additions and multiplications between the real components of complex numbers.
We use fixed-point integer representations,
so non-destructive additions add 1 bit of information
while multiplications can potentially double the amount of bits.
We wish to reduce the increase in resolution necessary to preserve all information.

The second issue is the simultaneous cross-updates
necessary for both complex multiplications and the FFT lattice.
These are theoretically reversible,
but we wish to develop a directly reversible algorithm.

\subsection{Lifting steps}

\subsection{Complex multiplication}

\subsection{Reversible convolution} % TODO: is 'convolution' the right word?

% TODO: Mention that the paper doesn't actually try to make directly reversible convolution

\begin{lstlisting}
x -= y
y <<= 1
y += x
\end{lstlisting}
